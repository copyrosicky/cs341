
Walmart.com maintains an online catalog of over 2M products. Consequently, enabling customers to quickly find products that conform to their specific needs and tastes is especially challenging. Given the difficulty of its task, Walmart.com's product search engine does an impressive job in interpretting the customer-provided query and rapidly returning relevant results. Yet, there remains highly significant information that is not fully leveraged. The details of a customer's online shopping session are indicative of a customer's intent and compliment--indeed, provide context for--the customer-provided query. In this report we describe and analyze a ranking scheme we call {\em Session Re-Rank} that can potentially induce a large increase in both click-through-rates and conversions on the first page of query results.

{\em Session Re-Rank} works by comparing previously clicked items with the top $N$ items returned by the search engine in reponse to a query. Items to be shown that are sufficiently similar to previously clicked items are promoted. The extent (i.e. number of positions) of the promotion for a particular item is a function of its similarity to previously clicked items, its original position, and the promotions of other items.

The similarity between an item to be shown and a previously clicked item is determined within five distinct vector spaces: {\em click-space}, {\em cart-space}, {\em query-space}, {\em title-space}, {\em item-space}. The non-unique representation of an item within each of these spaces may be thought of as a binary vector or a set of objects. (MapReduce jobs process historic query data to construct indexes whose keys are itemids and values are lists of the appropriate objects. Great care went into ensuring that index entries can be accessed in $\mathcal{O}(1)$ and that two entries can be merged to compute their intersection or union in linear time.) The similarity $J_s(A, B)$ of two items, $A$ and $B$, within a particular space $s$ is determined using Jaccard similarity. Similarities within particular spaces are then weighted and summed to determine the composite similarity
\begin{equation*}
S(A, B) = \sum{\limits{s}C_s(J_s(A, B))^{\alpha_s}},
\end{equation*}

where $C_s$ and $\alpha_s$ are tuning parameters. The score $\sigma$ attributed to an item to be shown is then the summation of composite similarities between itself and all previously clicked items plus the click-through-rate (CTR) $\Gamma_i$ of the item's original position $i$
\begin{equation*}
\sigma = \sum{\limits{B \in P}S(A, B)} + \Gamma_i,
\end{equation*}

where $P$ is the set of previously clicked items. 

The premise behind {\em click-space} is that two items are similar if they are both clicked within the same online shopping session. The dimensions, or objects, of this space are therefore past user-sessions. The {\em clicks-index} for the data presented in this report was contructed using approximately half of the Walmart provided data, or about 60M queries (about 120M page views).

{\em Cart-space} is based on the notion that two items are similar is they ever appear in a shopping cart together. The objects of this space are therefore shopping carts. The {\em clicks-index} for the data presented in this report was contructed using approximately half of the Walmart provided data.

Items are also considered similar if they appear in a query together. The objects of {\em query-space} are therefore queries. We make a distinction, however, between {\em user-queries} and {\em unique-queries}. The former are the well-defined entities within the raw Walmart data. The latter is an abstraction based on the notion that multiple {\em user-queries} can correspond to a single {\em unique-query}. To derive {\em unique-queries} from our data, we cluster {\em user-queries} as follows: two {\em user-queries} with the same search attributes (e.g. category or price filters) are considered the same {\em unique-query} if the strings constructed by concatenating the space-seperated, stemmed (we use the Python stemming.porter2 module), forced to lower-case, terms from each of their rawqueries are equal. We point out that while we achieved better results with this policy compared to simply using {\em user-queries}, we have no reason to believe that this is the ideal way to cluster queries for use within {\em Session Re-Rank}. Indeed, we believe one way to improve {\em Session Re-Rank} is to optimize the query clustering policy.

{\em Title-space} is straightfoward: each item is associated with a set of terms from its title. We ignore case, but at present do not stem, discard stop words, or weight terms in any way. 

Finally, the structure of {\em item-space} is unique because it involves a level of indirection. The premise here is that if items A and B are clicked in a single user-session and items A and C are clicked in another user-session, that items B and C are similar because they have item A in common.

